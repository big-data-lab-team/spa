\documentclass{IEEEtran}
\usepackage[hyphens]{url}
\usepackage{graphicx}
\usepackage{xcolor}
\usepackage{colortbl} % for \rowcolor
\usepackage{caption}
\usepackage{subcaption}
\usepackage[bookmarks=false]{hyperref}
\usepackage{hyperref}
\hypersetup{colorlinks=true,linkcolor=black,citecolor=blue,filecolor=black,urlcolor=blue}


\newcommand{\todo}[1]{\marginpar{\parbox{18mm}{\flushleft\tiny\color{red}\textbf{TODO}:
      #1}}}
\definecolor{headcolor}{gray}{0.9}

\begin{document}
\title{SPA: An Apache Spark Pilot scheduler for HPC applications}
\author{
    \IEEEauthorblockN{
        Val\'erie Hayot-Sasson and Tristan Glatard
    }
    \IEEEauthorblockA{}
}
\maketitle

\begin{abstract}
    \todo{complete}
\end{abstract}

\section{Introduction}
\section{Materials and Methods}\label{sec:methods}
    \todo{walltime parameter of workflow}
    \subsection{Added value of pilot scheduling}
        \todo{execution of a batch cluster vs dynamic (pilot) cluster}
    \subsection{Robust masters}
        \todo{scala standalone vs our pyspark workaround. kill masters in experiments}
    \subsection{Checkpointing}
        \todo{metric for determining how often to checkpoint based on cluster size}
    \subsection{Job arrays}
        \todo{need to kill idle workers. may not want all workers to be running at once.}
    \subsection{Example application}
        \todo{incrementation with varying task durations}
\section{Discussion}\label{sec:discussion}
\section{Conclusion}\label{sec:conclusion}

\end{document}
